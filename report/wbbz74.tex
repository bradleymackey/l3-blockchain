\documentclass[11pt]{article}
\renewcommand{\familydefault}{\sfdefault}

% Make margins smaller
\usepackage[top=1in, bottom=1in, left=1in, right=1in]{geometry}
% more advanced mathematical symbols
\usepackage{amsfonts}
\usepackage{amssymb}
\usepackage{amsmath}
\usepackage{bm}
\usepackage{bold-extra} % bold texttt
\usepackage{graphicx}
\usepackage{enumerate}
\usepackage[font=small,labelfont=bf]{caption} % Required for specifying captions to tables and figures
\usepackage{pgfgantt} % gantt chart!
\usepackage{titling}
\usepackage{setspace}
\renewcommand{\baselinestretch}{1.2} 
\usepackage{hyperref}
\usepackage{seqsplit}

\expandafter\def\expandafter\UrlBreaks\expandafter{\UrlBreaks%  save the current one
  \do\a\do\b\do\c\do\d\do\e\do\f\do\g\do\h\do\i\do\j%
  \do\k\do\l\do\m\do\n\do\o\do\p\do\q\do\r\do\s\do\t%
  \do\u\do\v\do\w\do\x\do\y\do\z\do\A\do\B\do\C\do\D%
  \do\E\do\F\do\G\do\H\do\I\do\J\do\K\do\L\do\M\do\N%
  \do\O\do\P\do\Q\do\R\do\S\do\T\do\U\do\V\do\W\do\X%
  \do\Y\do\Z}

\graphicspath{ {./images/} }

% so that we can skip any number of items in an enumeration
\makeatletter
\newcommand{\skipitems}[1]{%
  \addtocounter{\@enumctr}{#1}%
}
\makeatother

% ALL OF THE MARKS FOR THE ASSIGNMENT COME FROM THE REPORT!

\begin{document}

\title{\textbf{Blockchain Summative}}
\date{for 22nd March 2019}
\author{Bradley Mackey}
\maketitle

Please be aware that when copying and pasting hashes broken across multiple lines, \textbf{the pasted output may include a space character at the line break}, please account for this.


\section*{Task 1 - Mining Puzzles}

\begin{enumerate}
\item \textit{User ID}: \textbf{\texttt{wbbz74}}
\item \textit{Block hash target}: \texttt{\seqsplit{000003e7fc180000000000000000000000000000000000000000000000000000}}
\item \textit{Valid nonce}: \textbf{\texttt{3856645}}
\item \textit{Number of double hashes}: \texttt{3856645}\\\textit{Time taken}: \textbf{103.41s}
\item 
\textit{Time to mine at initial difficulty of $1$}\\
Difficulty ($D$) 0.001 takes \texttt{103.41s}\\
$D=1 \implies 103.41\times{\frac{1}{0.001}}=103,410$ seconds\\
$\implies \frac{103410}{60}=1,723.5$ minutes\\
$\implies \frac{1723.5}{60}=$ \textbf{28.72 hours.}\\
\\
\textit{Time to mine at peak 2018 difficulty of $7,454,968,648,263$}\\
$D=7454968648263 \implies 103.41\times{\frac{7454968648263}{0.001}}=7.70918\times10^{14}$ seconds\\
$\implies \frac{7.70918\times10^{14}}{60\times60\times24}=8.92266\times10^{9}$ days\\
$\implies \frac{8.92266\times10^{9}}{365.25}=24,428,927.04$ years\\
$\implies \frac{24428927.04}{1000}=$ \textbf{24,428.9 millenia.}\\

\item \textit{ECDSA Public Key}: \texttt{\seqsplit{14afbb92502c9294f19be099ac3fe51f8ea1c943e36a06c43b096864d887145b55e87f1a01b1b9275bcc9d528a2829a774ec6de06dfaed72933ced851105f3ba}}

\item \textit{Hello World Signature}: \texttt{\seqsplit{acd855318df6ebb70e4c956caad1c7df1a3395c2ead557e6ec304ced9038037aa83e79ab1bb80ca3b912ea2806c67cc387301f1530e730834bb3213cf55b70d6}}

\item \textit{Signed previous Generation Signature ($SK(G)$)}: \texttt{\seqsplit{262ebf16d77c247020ad0b37ac133475c50cf273c81b592370be51fcca05ee1b68f4dd5cded105bf7ea4e3823dea2766f0c46452177805c39207371192184fd7}}

\item \textbf{\textit{Hit Value}}\\
SHA256(SK($G$)) = \texttt{\seqsplit{0474501e05347f67cf3169f1ac64638c460252e91b7c8f9b1aa880d593754455}}\\
$\therefore$ Hit Value (first 8 bytes) = \textbf{\texttt{\seqsplit{0474501e05347f67}}}

\item \textbf{\textit{Time to forge new block}}\\
Effective balance ($E$) = 74\\
Base target ($T_B$) = 1229782938247303\\
Time since last block ($t$) = $t$ (to determine)\\
$\therefore$ New Target = $E\times T_B\times t=91003937430300422\times{t}$\\
\\
Block can be forged when hit value is less than target.\\
Hit Value (decimal) = 320969563316715367\\
$\therefore t=\frac{320969563316715367}{91003937430300422}=3.527$ (3 d.p.)\\
$\therefore$ \textbf{Block can be forged after 3.527 seconds.}\\
\begin{small}
As the signature varies each time the previous signature is signed, this value will vary as the hit value varies. After a few seconds of trying values, this was the shortest value I was able to find. Finding an optimal signature and then using this value to forge the block is optimal as it means that more block rewards will able to be claimed.
\end{small}


\end{enumerate}

\section*{Task 2 - Transactions \& bitcoin-testnet}

\begin{enumerate}
\item User ID: \texttt{wbbz74}
\item 
\begin{enumerate}
\item \url{https://www.blockchain.com/btc/tx/cfe6cc5158f435f59c4daa24f66378ff56baf2980d04c92612e2adf222bb19b8}
\item \url{https://www.blockchain.com/btc/tx/348e8846eccca909c67eade94b3df0c84ab07133159b25759f4b3cac303904ec}
\item \url{https://www.blockchain.com/btc/tx/f3d7d00d0534fd7d59fb1cb4311dad4e42fef1b6174321342a9ed2af21d9bd25}
\end{enumerate}
\item 
\begin{enumerate}

\item This is a transaction with 2 inputs and 3 outputs. One of the outputs is a basic zero-value data transaction, making use of the \texttt{OP\_RETURN} word to ensure the output can never be redeemed, placing some arbitrary data into the blockchain. 
The 2 inputs, as well as the 2 remaining outputs, use a Pay-to-Public-Key Hash (P2PKH) scheme for transferring coins---this can be identified as all the addresses begin with the number \texttt{1}. 
The inputs prove to the blockchain they are in control of the private keys associated with the previous transaction that sent them the coins by providing a signature---derived from their private key---and their public key. 
At the end of execution of the script, if the signatures are valid, the stack terminates with \texttt{TRUE}. This allows the inputs to be sent successfully. 
The recipients, also using P2PKH, provide a script which will allow them to later redeem coins in a transaction block, given they are in posession of the private key for the recipient address. This scriptPubKey (the script used to lock the Bitcoins) is of the form: \texttt{OP\_DUP} \texttt{OP\_HASH160} \textit{hashedPublicKey} \texttt{OP\_EQUALVERIFY} \texttt{OP\_CHECKSIG}.

\item This is a simple transaction with 1 input and 2 outputs. The input uses a P2PKH scheme similar to the previous transaction. One of our outputs also uses P2PKH, but the other uses Pay-to-Script-Hash (P2SH)---which can be identified as the address begins with a \texttt{3}. This differs from most other standard scripts available on the Bitcoin network as it allows for any arbitrarily complex script to take constant space on the blockchain, taking only 23 bytes. As the Bitcoin is being sent to a P2SH scheme, little effort is required, the sender only has to check the hash of what is otherwise a very long script (for example, this could be a large multi-sig transaction, but we are unaware of this). P2SH's scriptPubKey is of the (much shorter) form: \texttt{OP\_HASH160} \textit{hashedScript} \texttt{OP\_EQUAL}.

\item This is a transaction with 2 inputs and 3 outputs. One of the outputs is a zero-value data transaction. One of the inputs and outputs use P2PKH. This transaction also includes an input and output using Pay-to-Multisig (P2MS). These have the same requirement of requiring only 1 possible signature for 3 possible public keys, enabling 3 keyholders to exist, but only 1 of these is needed to authorise the sending of coins. Focusing on the input, we see the scriptPubKey is of the form: \texttt{OP\_1} \textit{pubKey1} \textit{pubKey2} \textit{pubKey3} \texttt{OP\_3} \texttt{OP\_CHECKMULTISIG}. The scriptSig (script used to unlock the coins, to be sent in a transaction) is of the form: \texttt{OP\_0} \textit{signature}. It is within this script we see the single signature required in order to send the previously locked Bitcoins. The reason for the \texttt{OP\_0} before the signature is due to the well known bug in \texttt{OP\_CHECKMULTISIG}, where the signature extraction variable (\texttt{OP\_1} in our case) will consume 1 more input than stated. Therefore, we pad the stack with a dummy `\texttt{OP\_0}' to prevent this function consuming data it is not supposed to.

\end{enumerate}

\item \textit{Bitcoin Testnet Address}: \texttt{mjLjznCbyKuGJ5xuz7Wo1Es3qXHoxoDXgo}

\item \textbf{\textit{100 Satoshi Transaction}}\\
\textit{TX ID}: \texttt{\seqsplit{74b5486e061ac680cde0f132b0dec6c5010d2dee8da3a2856d680fcf5bf41c37}}\\
\textit{Link}: \url{https://chain.so/tx/BTCTEST/74b5486e061ac680cde0f132b0dec6c5010d2dee8da3a2856d680fcf5bf41c37}

\item \textbf{\textit{Student ID Proof-of-Burn Transaction}}\\
\textit{TX ID}: \texttt{\seqsplit{bd1c2552fc0effda71e4e09137d8106aa6c67239dfba1e760040d1c78b66e0ac}}\\
\textit{Link}: \url{https://chain.so/tx/BTCTEST/bd1c2552fc0effda71e4e09137d8106aa6c67239dfba1e760040d1c78b66e0ac}

\item \textbf{\textit{Student ID Proof-of-Burn Script}}\\
\textit{Script Hex}: \texttt{6a067762627a3734}\\
We can add data to the blockchain by immediately invalidating the script, allowing the remainder of the script to be interpreted as pure data.
The first byte, \texttt{6a}, is the \texttt{OP\_RETURN} word. This invalidates the script, such that any attempt to redeem any Bitcoins contained in this transaction would instantly fail, as per the semantics of Bitcoin Script (therefore this is a very bad script to use if actually sending bitcoins!).
The next byte, \texttt{06}, is the number of bytes that we will push onto the stack next---``wbbz74'' is 6 characters long (6 bytes when ASCII encoded), so this is just 6. 
The remainder of the script, \texttt{7762627a3734}, is the ASCII encoded ``wbbz74'', which will be interpreted on the blockchain as pure data.

\end{enumerate}

\section*{Task 3 - Wise Investments}

% In your report explain in 400-500 words whether you think it would be wiser to invest in bitcoin, NXT or gold (real physical gold) at the moment.
% Justify your stance as far as possible.
% (I am not actually going to judge your investment advice, but rather whether you have considered all the different factors, suitably evaluated them and come to a coherent conclusion.)

The historical value, scarcity and reputation of gold make it difficult to criticise as an investment. It is \textit{so} stable as a means of maintaining value, a huge number of countries posses thousands of tons in reserves\footnote{\url{https://en.wikipedia.org/wiki/Gold_reserve}}.

A large factor contributing to the value of gold is its inherent scarcity. Bitcoin shares this feature, having the \textit{block reward} halve every 210,000 blocks---meaning no more than 21 million Bitcoins will ever be minted. Possessing some of these 21 million coins gives you a `share' in Bitcoin that you know will never be diluted, much like gold.

However, Bitcoin's value is largely upheld by the fact it is a decentralised medium of exchange---currency that cannot be transferred easily is of little to no use. To transact Bitcoin to third-party, this transaction must be included in a block created by a miner, which could be anyone who chooses to become one. Bitcoin mining is an \textit{incredibly} power hungry task, and miners are only incentivised to keep mining by the price of Bitcoin itself---as that is how miners are rewarded, and is the only factor offsetting the cost of power. Should the price of Bitcoin fall dramatically, miners would be less incentivised to continue and the transaction throughput rate would reduce until the difficulty re-adjusts. Depending on the severity of this drop, transactions could be essentially frozen for months or years---this could cause a loss of faith in the system.

Nxt also has a fixed supply of currency (1 billion tokens) but does not suffer from the power draw problem. Instead of mining, the largest stakeholders are the actors who can probabilistically create transaction blocks based on their stake in the network. As the creation of blocks is so cheap, there is a much lower change of the network becoming frozen in a hypothetically similar way to Bitcoin. Although, these largest stakeholders have a huge amount of control over network, it is decidedly less decentralised than Bitcoin, as a very few people have very large amounts of control over the transactions---smaller players are only able to create blocks much after these largest stakeholders have a chance. In terms of functioning as a currency, it is far more likely Nxt will still be tranacting after a future possible `crypto depression' than Bitcoin.

Bitcoin's prominence makes it the gold standard of modern cryptocurrencies, with the market cap of all other cryptocurrencies being essentially relative to it. This also makes all cryptocurrencies highly volatile, as people figure out if they a force for good or evil.

Nxt is clearly an improvement over Bitcoin for the reasons set out. However, for it to maintain value and relevance, people need to \textbf{believe} in it. The huge number of available cryptocurrencies and their various advantages/disadvantages make Nxt blend too well---it would be a highly risky investment.

Due to the continuing volatility in the market price of Bitcoin and rise of other innovate alt-coins (which could potentially de-throne Bitcoin), I can only truly recommend an investment in gold. 

\end{document}

